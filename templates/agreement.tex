\documentclass[a4paper, parskip=half]{scrlttr2} 

\usepackage[utf8]{inputenc} 
\usepackage[T1]{fontenc} 
\usepackage[ngerman]{babel} 
\usepackage{lmodern} 
\usepackage[juratotoc]{scrjura}
\usepackage[a4paper,margin=1.5cm,top=1.6cm,bottom=4cm]{geometry}
\usepackage{lastpage}
\usepackage{fancyhdr}% headers
\usepackage{graphicx} % logo include
\usepackage{color} % text colors
\usepackage{setspace}
\usepackage{enumitem,amssymb}
\newlist{todolist}{itemize}{2}
\setlist[todolist]{label=$\square$}
\usepackage{pifont}
\newcommand{\cmark}{\ding{51}}%
\newcommand{\xmark}{\ding{55}}%
\newcommand{\done}{\rlap{$\square$}{\raisebox{2pt}{\large\hspace{1pt}\cmark}}%
\hspace{-2.5pt}}
\newcommand{\wontfix}{\rlap{$\square$}{\large\hspace{1pt}\xmark}}

\usepackage{paratype} % font
\renewcommand*\familydefault{\sfdefault} %% Only if the base font of the document is to be sans serif
\usepackage[T1]{fontenc}

\definecolor{color-dark}{RGB}{50,50,50}
\definecolor{color-grey}{RGB}{100,100,100}





% header
\pagestyle{fancy}
\fancyhf{}
\fancyhead[R]{\vspace{5pt} \includegraphics[width=5cm]{C3W_RB.png}\vspace{-1.2cm}}
\fancyhead[L]{\textcolor{color-grey}{\textbf{Chaos Computer Club Wien (C3W)}\\\small buero@c3w.at}
}
\fancyfoot[C]{\fontsize{7}{7} Chaos Computer Club Wien (C3W)\\ \href{https://c3w.at/}{https://c3w.at/}}


% lines
\renewcommand{\headrulewidth}{2pt}
\renewcommand{\headrule}{\hbox to\headwidth{%
  \color{color-grey}\leaders\hrule height \headrulewidth\hfill}}
\renewcommand{\footrulewidth}{0.5pt}
\renewcommand{\footrule}{\hbox to\headwidth{%
  \color{color-grey}\leaders\hrule height \footrulewidth\hfill}}
\color{color-dark}




\usepackage[T1]{fontenc}
\usepackage{selinput}
\SelectInputMappings{adieresis={ä},germandbls={ß}}
\usepackage[ngerman]{babel}

\usepackage{tabularx}
\usepackage{booktabs}

\usepackage{numprint}
\nprounddigits{2}

\usepackage[right]{eurosym}

\usepackage{datatool}
\DTLnewdb{talkliste}

\newcommand\Talk[2]{%
  \DTLnewrow{talkliste}
  \DTLnewdbentry{talkliste}{Bezeichnung}{#1}
  \DTLnewdbentry{talkliste}{Aufzeichnung}{#2}
}

\newcommand\einreichungtxt{}
\newcommand\aufzeichnungtxt{}
\newenvironment{talkliste}[2]{%
  \DTLcleardb{talkliste}%
  \renewcommand\einreichungtxt{#1}%
  \renewcommand\aufzeichnungtxt{#2}%
}{%
  \par\centering
  \begin{tabularx}{\textwidth}{Xrrr}
    \toprule
    \einreichungtxt&\aufzeichnungtxt
    \DTLforeach*{talkliste}{\bez=Bezeichnung,\aufz=Aufzeichnung}{%
      \DTLiffirstrow{\\\midrule}{\\}
      \bez&\aufz
      \vspace{0.2 cm} 
    }\\
    \bottomrule
  \end{tabularx}
  \par
}

\renewcommand\raggedsignature{\raggedright}
\usepackage{hyperref}



\newcommand{\speakername}{}

\begin{document} 
%- for speaker in speakers
\clearpage
\thispagestyle{fancy}
 
 
\renewcommand{\speakername}{\VAR{speaker.public_name|e}}
\fancyfoot[R]{\textcolor{color-grey}{\speakername}}
 
\vspace*{0.4 cm}
\centerline{\includegraphics[width=8cm]{PW19_Logo.pdf}}

\vspace*{0.5 cm} 
\centerline{\LARGE{\textbf{Speaker’s Agreement}}}
\vspace*{0.25 cm} 
\centerline{\Large{PrivacyWeek 2019}}

\vspace{1cm}
\large

%- if speaker.language == 'de'

\begin{todolist}
%- if not speaker.record
    \item[\done] Meine Vorträge bzw. Workshops (unten genannt) werden nicht aufgezeichnet und sind weder im Livestream, 
        noch in den Veröffentlichungen zu sehen.
%- else
    \item[\done] Ich bin einverstanden mit Streaming und Aufzeichnung der unten genannten Vorträge in Bild, Ton und Text. \\
    Die Aufzeichnungen meines Vortrags dürfen von der PrivacyWeek veröffentlicht und zum Download bereitgestellt werden. \\
    Auch der Livestream darf öffentlich übertragen werden.
    \item Die Aufzeichnungen meines Vortrags dürfen von der PrivacyWeek für eventbezogenes Werbematerial verwendet werden.
%- endif
\end{todolist}

\vspace{1.1cm}

\begin{tabular}{p{7cm}p{.5cm}l}
\dotfill \\ 
Ort, Datum
\end{tabular}
\hfill 
\begin{tabular}{p{7cm}p{.5cm}l}
\dotfill \\ 
\speakername
\end{tabular}

\vspace{0.5 cm}

\normalsize
Weitere Informationen zum Datenschutz des C3W: \href{https://c3w.at/datenschutz}{c3w.at/datenschutz}

\vfil

\begin{talkliste}{Einreichung}{Aufzeichnung}
%- for talk in speaker.talks|sort(attribute='do_not_record')
  \Talk{\VAR{talk.title|e}}{
%- if not talk.do_not_record
    \cmark \qquad
%- else
    \xmark \qquad
%- endif
}
%- endfor
\end{talkliste}
\vspace{0.1cm}
Etwaige Verstöße gegen die Rechte (z.B. Urheberrechte) Dritter liegen in der Verantwortung der jeweiligen Vortragenden.
\vfil

%- else

\begin{todolist}
%- if not speaker.record
    \item[\done] My talks, workshops or art performances are not going to be recorded by the PrivacyWeek team and won't be live-streamed or published afterwards.
%- else
    \item[\done] I agree with the streaming, recording, taking photos and textual documentation of my talks, workshops or art performances. \\
        The recordings may be published and provided for download by C3W and the PrivacyWeek team. \\
        Also the Livestream may be broadcasted publicly.
    \item The recordings of my talk, workshop or art performance may be used by C3W and the PrivacyWeek team for 
        event related advertising material (e.g. trailer).
%- endif
\end{todolist}


\vspace{1.1cm}

\begin{tabular}{p{7cm}p{.5cm}l}
\dotfill \\ 
Place, Date
\end{tabular}
\hfill 
\begin{tabular}{p{7cm}p{.5cm}l}
\dotfill \\ 
\speakername
\end{tabular}

\vspace{0.5 cm}

\normalsize
Further information on C3W data privacy: \href{https://c3w.at/datenschutz}{c3w.at/datenschutz}

\vfil

\begin{talkliste}{Title}{Recording}
%- for talk in speaker.talks|sort(attribute='do_not_record')
  \Talk{\VAR{talk.title|e}}{
%- if not talk.do_not_record
    \cmark \qquad
%- else
    \xmark \qquad
%- endif
}
%- endfor
\end{talkliste}
\vspace{0.1cm}
Any infridgements of third-party rights in talks, workshops or performances fall solely in the responsibility of their speakers.
\vfil

%- endif

%- endfor
\end{document}
